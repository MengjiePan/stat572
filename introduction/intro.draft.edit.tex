\documentclass{uwstat572}

%%\setlength{\oddsidemargin}{0.25in}
%%\setlength{\textwidth}{6in}
%%\setlength{\topmargin}{0.5in}
%%\setlength{\textheight}{9in}

\renewcommand{\baselinestretch}{1.5} 


\bibliographystyle{plainnat}

\usepackage{color}
\usepackage{ulem}
\newcommand{\vmdel}[1]{\sout{#1}}
\newcommand{\vmadd}[1]{\textbf{\color{red}{#1}}}
\newcommand{\vmcomment}[1]{({\color{blue}{VM's comment:}} \textbf{\color{blue}{#1}})}

\begin{document}
%%\maketitle



\begin{center}
  {\LARGE Forecasting Time Series With Complex Seasonal Patterns Using Exponential Smoothing}\\\ \\
  {Mengjie Pan \\ 
    Department of Statistics, University of Washington Seattle, WA, 98195, USA
  }
\end{center}



%\begin{abstract}
%  Put your project summary here.
%\end{abstract}

\section{Introduction}

\hspace{4ex}Many time series have complex seasonal patterns, such as non-integer seasonal periods, multiple nested and non-nested seasonal periods. For example, a weekly series of U.S. gasoline products supplied has an annual seasonal pattern with a non-integer period of 365.25/7 $\approx 52.18$. A five-minute interval series of number of call arrivals in a commercial bank has a daily seasonal pattern and a weekly seasonal pattern, which is an example of multiple nested seasonal periods. We also provide an example of non-integer and multiple non-nest seasonal periods: a daily series of the electricity demand in Turkey is affected by events on the Hijri calendar of a period of 354.37 and the Gregorian calendar of a period of 365.25, and therefore has two annual seasonal patterns whose periods are approximately eleven days apart. 

A number of existing \vmdel{work} \vmadd{approaches} deal with some of the above complexities. \citet{pedregal2006modulated}, \citet{harvey1993forecasting} and \citet{taylor2003short} considered double seasonality, and \citet{taylor2010triple} considered triple seasonality. \citet{harvey1997modeling} used single trigonometric seasonality to deal with non-integer seasonal period. However, none of the previous \vmdel{work} \vmadd{methods} allows \vmadd{for} non-integer seasonal periods and multiple seasonal periods at the same time, and therefore cannot be applied to Turkish electricity demand time series. To \vmdel{this end} \vmadd{remedy this shortcoming}, \vmadd{\citet{de2011forecasting}} \vmdel{we} propose two new innovations state space models for exponential smoothing, and one of them is based on trigonometric formulation to incorporate non-integer seasonal period. 

\vmcomment{The above and below paragraphs need better a better connection/transition.}
Simple exponential smoothing calculates forecasts using weighted averages of observed time series, where the weights decrease exponentially as observations becomes further in the past. \citet{holt1957forecasting} and \citet{winters1960forecasting} extended simple exponential smoothing to include trend and seasonality components in the model. \citet{hyndman2002state} discussed a class of innovations state space models, where `innovations' refers to \vmadd{a} single source of error. The innovations state space framework allows trend, seasonality, error specification of forecasting models, and the Holt-Winters additive and multiplicative methods are specific cases of the framework with an additive trend. One of the most commonly used single seasonal models in the framework is the linear version of the Holt-Winters method with additive errors:
\begin{eqnarray}
y_t &=& l_{t-1}+b_{t-1}+s_{t-m_1}^{(1)}+d_t,\\
l_t &=& l_{t-1}+b_{t-1}+\alpha d_t,\\
b_t &=& b_{t-1}+\beta d_t,\\
s_t^{(1)} &=& s_{t-m_1}^{(1)}+\gamma d_t,
\end{eqnarray}
where $\vmadd{\{}y_t\vmadd{\}}$is the time series, $m$ is the period of the seasonal pattern, $\{d_t\}$ is a series of uncorrelated white-noise random variable, $l_t, b_t, s_t$ represent the level, trend and seasonal components at time $t$, and $\alpha,\beta,\gamma$ are smoothing parameters.

One problem is that this model does not allow \vmadd{for} non-integer periods and multiple seasonal periods. More seasonal components can be added to this model\vmadd{,} ($s_t^{(2)},...,s_t^{(T)}$); however, a large number of initial states ($l_0, b_0, \{ s^{(1)}_{1-m_1}, ..., s_0^{(1)} \},...,  \{ s^{(T)}_{1-m_T}, ..., s_0^{(T)} \}$) need to be estimated when one of the seasonal patterns has large periods. \citet{taylor2003short}, \citet{taylor2010triple}, and \citet{gould2008forecasting} approximated initial seasonal values, but their methods did not lead to optimized values. \vmadd{--- I didn't understand this sentence}
Another problem is that the above model assumes that the white noise process\vmadd{es?} are uncorrelated, but \vmadd{there is} empirical\vmdel{ly} \vmadd{evidence that} the error process\vmadd{es?} might be correlated \citep{taylor2003short}. 
A further problem is that the nonlinear version of state space models can have infinite forecast variances \citep{akram2009exponential}. 

To accommodate non-integer seasonal periods and multiple seasonal periods, \vmadd{\citet{de2011forecasting}} \vmdel{we} propose two innovations state space models for exponential smoothing (BATS and TBATS \vmcomment{--- define these abbreviations}), \vmadd{where} \vmdel{and} TBATS is \vmadd{an extension of BATS} based on trigonometric formulation. With trigonometric formulation, the number of initial states estimated significantly decreases, and our estimation algorithm returns optimized \vmcomment{the word optimized is meaningless if you don't state the optimization criterion; do you mean estimated here?} least squares estimates of initial states and maximum likelihood estimates of parameters. \vmadd{\citet{de2011forecasting}} \vmdel{We} model the error process $\{d_t\}$ as an ARMA process to allow correlation of errors. \vmadd{\citet{de2011forecasting}} \vmdel{We} use a Box-Cox transformation to allow some nonlinearity in our model, while avoiding the problems with nonlinear models mentioned above. \vmadd{\citet{de2011forecasting}} \vmdel{We} introduce a new state space modeling framework, which includes the two new models we propose. \vmcomment{Your writing became too monotonic in the last three sentences.}

With estimated parameters and initial states, \vmadd{\citet{de2011forecasting}} \vmdel{we} can obtain point forecasts and forecast intervals using inverse Box-Cox transformation of quantiles of the prediction distribution. 
\vmdel{Our} \vmadd{We reproduced the} proposed models are applied \vmadd{them} to all three datasets of U.S. gasoline supply, number of call arrivals, and electricity demand in Turkey.  We evaluate our forecasts from our proposed models using the root mean squared error (RMSE), and we compare BATS forecasts with TBATS forecasts.

\nocite{*}

\bibliography{ref}


\end{document}









